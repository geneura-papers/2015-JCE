%%%%%%%%%%%%%%%%%%%%%%%%%%%%%%%%  ejemplo.tex %%%%%%%%%%%%%%%%%%%%%%%%%%%%%%%

%%%%% Fichero de ejemplo LaTeX que ilustra el uso de la Hoja de Estilo %%%%%%
%%%%% Jornadas.cls para Jornadas Sarteco.              %%%%%%

\documentclass[twocolumn,twoside]{Jornadas}
\usepackage[latin1]{inputenc}
\usepackage[dvips]{epsfig}
\usepackage{url}

\def\BibTeX{{\rm B\kern-.05em{\sc i\kern-.025em b}\kern-.08em
    T\kern-.1667em\lower.7ex\hbox{E}\kern-.125emX}}

\newtheorem{theorem}{Teorema}



%%%%%%%%%%%%%%%%%%%%%%%%%%%%%%%%%%%%%%%%%%%%

\hyphenation{}

\begin{document}


\title{Arquitectura de monitorizaci�n de veh�culos usando BLABLABA}

\author{%
     Pablo Garc�a S�nchez, Juan Juli�n Merelo, Pedro Castillo, A. M. Mora%
     \thanks{Dpto. de Arquitectura y Tecnolog�a de Computadores, Universidad de Granada, e-mail: {\tt {pgarcia,jjmerelo,pedro,amorag}@atc.ugr.es}}
     Rub�n H. Garc�a, Miguel �ngel L�pez, Mar�a Isabel L�pez%
     \thanks{Fundaci�n I+D del Software Libre, Granada, e-mail: {\tt {malopez,rhgarcia,milopez}@fidesol.org}}
}

\maketitle
% Oculta las cabeceras y los n\'umeros de p\'agina.
% Ambos elemetos se a\~nadir\'an durante la edici\'on de las actas completas.
\markboth{}{}
\pagestyle{empty} 
\thispagestyle{empty} % Oculta el n\'umero de la primera p\'agina

\begin{abstract}

\end{abstract}

\begin{keywords}
Bluetooth, Predicci�n del Tr�fico, Monitorizaci�n, WiFi
\end{keywords}

\section{Introducci\'on}
\PARstart{E}{s} un hecho innegable que nos encontramos \cite{Castillo13Indicadores}

\section{Estado del arte}

\section{Descripci�n del sistema}



\begin{figure*}[ht] 
\begin{center} 
%\includegraphics[scale=0.35]{monitorNT}
%\epsfig{file=images/sindicador.eps,width=12cm}

\end{center} 
\caption{Subsistema sindicador. Este sistema permite al usuario administrar el servidor y recibir informaci�n por el canal de retorno.} 
\label{fig:sindicador} 
\end{figure*}

\section{Conclusiones}



\section*{Agradecimientos}


\bibliographystyle{Jornadas}
\bibliography{monitorizacion-jce}

\end{document}
